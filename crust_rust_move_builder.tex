\documentclass[8pt]{beamer}

\mode<presentation> {
    \usetheme{Warsaw}}
\usepackage{tikz}
\usepackage{bbm}
\usepackage[utf8]{inputenc}
\usepackage{amsmath, amsthm, amssymb}
\usepackage{graphicx}
\usepackage{enumerate}
\usepackage{mathrsfs}
\usepackage{mathtools}
\usepackage{bookmark}
\usepackage{tikz-cd}
\usepackage{svg}
\usepackage[cache=false]{minted} 
%\newtheorem{theorem}{Theorem}
\newtheorem{prop}{Proposition}
%\newtheorem{lemma}{Lemma}
%\newtheorem{definition}{Definition}
%\newtheorem{example}{Example}
\newtheorem{question}{Question}
\theoremstyle{remark}
\newtheorem*{remark}{Remark}
\newcommand\floor[1]{\lfloor\#1\rfloor}
\DeclareMathOperator*{\argmax}{argmax}
\DeclareMathOperator*{\argmin}{argmin}
\title {The Crust of Rust -- Move Semantics, Initialization, and the Builder Pattern}
\author{Varoon Pazhyanur}
\institute{Bloomberg L.P.}
\date{13 November 2020}
\usepackage{hyperref}
\hypersetup{
        colorlinks=true,
        linkcolor=red,
        filecolor=magenta,      
        urlcolor=blue,
    }
\begin{document}
    \begin{frame}
        \titlepage 
    \end{frame}
    \begin{frame}[fragile]
        In CPP the \href{https://isocpp.github.io/CppCoreGuidelines/CppCoreGuidelines\#S-ctor}{Rule of 5} (or 0,3,6,7, etc. depending on whom and when one
        asks) is the lifeblood of RAII and how CPP relates objects and memory.
        The CPP Core guidelines dedicate many subsequent clarifications. For
        example, one should prefer the compiler-generated defaults unless there
        is a good reason to do otherwise (C.20), IN particular, the destructor
        (C.37) and move operations (C.66) should be noexcept. Every class X
        must have the following:

        \begin{minted}[frame=single]{cpp}

            class X{
                X(const X&)=default;
                operator=(const X&)=default;
                X(X&&)=default;
                X& operator=(X&&)=default;
                ~X()=default; 
            }
        \end{minted}
        The Rust analog of the destructor is the Drop trait, which we ignore
        for now. The two ampersands denote an r value reference, a language
        construct introduced in CPP11 to tag a reference as being temporary in
        some sense. The move constructor and move assignment operators leave
        the input in a valid but unspecified state, meaning that all class
        invariants must be preserved. In particular, move semantics typically
        involves moving identities (pointers and references) or scopes, rather
        than values (I try to clear my data structures explicitly after moving
        out of them). Copy constructing a set, for example, might invoke a copy
        for every node. Move constructing a set, by contrast, may only involve
        calling std::swap on nullptr and a pointer to the root node of a tree
        (plus some minor clean up to preserve set invariants). 
    \end{frame}
    \begin{frame}
        \frametitle{Motivating Examples}
        Our code frequently chains operations on data structures especially
        with unary and binary functions, so we can focus on those. E.x.
        Pandas dataframes, matrices/ndarrays, sets, maps. While copying an
        integer may be ok, needlessly copying entire matrices or dataframes
        can kill performance. CPP and Rust in particular have relatively
        thorough and high level semantics for explicitly controlling when
        to copy and when to "steal" an data structure's contents. Rust's
        ownership sustem helps to avoid common bugs and adhere to many of
        the CPP Core Guidelines.   
        \begin{enumerate}
            \item Consider the operation of merging/unioning two sets, both
                ordered or unordered. As of CPP17, one can merge a set into
                another set by copying or moving.
                \href{https://en.cppreference.com/w/cpp/container/set/merge.}{std::set.merge}. The
                merge method on std::map is similar.  
            \item One can move to \href{https://en.cppreference.com/w/cpp/container/vector/push_back}{push pack} into a vector
                 or
                 \href{https://en.cppreference.com/w/cpp/container/vector/insert}{insert} many elements into a vector. See
                overload 2.  
            \item std::move is a static cast to an r value reference and can
                frequently be used to force move semantics.
            \item Lambdas can use move semantics as of CPP14 with the help of
                std::move.
        \end{enumerate}
    \end{frame}
    \begin{frame}
            \frametitle{Move Semantics in Rust}
            In Rust, function parameters move by default, mimicing passing by r
            value reference from CPP. While it was a CPP convention to never
            access a variable after it has been moved out of, the borrow
            checker turns using an identity after it has been moved a compile
            time error. This permits safe, frequent move semantics. See, for
            example, 
            \begin{enumerate}
                \item \href{https://docs.rs/hyper/0.13.9/hyper/server/struct.Builder.html}{hyper::server::Builder} v0.13.9
                \item \href{https://docs.rs/reqwest/0.10.9/reqwest/struct.RequestBuilder.html}{reqwest::RequestBuilder} v0.10.9                
                \item \href{https://docs.amethyst.rs/stable/amethyst/struct.GameDataBuilder.html}{amethyst::GameDataBuilder} v0.15.3
                \item \href{https://docs.rs/rayon/1.5.0/rayon/struct.ThreadPoolBuilder.html}{rayon::ThreadPoolBuilder} v1.5.0
                \item arrow::csv::reader::ReaderBuilder v2.0.0
                https://docs.rs/arrow/2.0.0/arrow/csv/reader/struct.ReaderBuilder.html
                \item clap::App v2.33.3
                \href{https://docs.rs/clap/2.33.3/clap/struct.App.html}{text}
                \item mysql::OptsBuilder v20.1.0
                \href{https://docs.rs/mysql/20.1.0/mysql/struct.OptsBuilder.html}{text}
            \end{enumerate} 

            Other important uses of move semantics in Rust include the folliwng.

            \begin{enumerate}
                \item The primary conversion traits std::convert::From \href{https://doc.rust-lang.org/std/convert/trait.From.html\#tymethod.from}{text} and std::convert::TryFrom \href{https://doc.rust-lang.org/std/convert/trait.TryFrom.html\#tymethod.try_from}{text}. 
            \end{enumerate}
    \end{frame}
\end{document}